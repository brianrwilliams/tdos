\section{The Formal Theory}


\owen{This section should also have an introduction that indicates how it is parallel to the purely algebraic construction just described.}

We construct one-dimensional Chern-Simons \owen{theory} with target the formal
disk. We will use this theory to study how diffeomorphisms on the target act
on the theory, which will be encoded by the structure of
$\Vect$-equivariance. 
\owen{This second sentence is befuddling.}

We have already seen \owen{?} that the formal disk is encoded by the dg Lie
algebra $\fg_n = \CC^n[-1]$. Consider the dg Lie algebra 
\ben
\fg_n^\RR := \Omega^*(\RR ; \fg_n) .
\een  
This dg Lie algebra is abelian and the differential is the de Rham
differential $\d_{dR}$. This dg Lie algebra encodes deformations of the
constant map to 0 to nearby smooth functions. 

The dg Lie algebra that encodes one-dimensional Chern-Simons with
target $\hD^n$ is defined to be the ``double''
\ben
\DD \fg_n^\RR = \Omega^*(\RR ; \fg_n \oplus \fg_n^\vee[-2]) .
\een 
The Lie bracket is still trivial, and the differential is
$\d_{dR}$. This dg Lie algebra encodes the cotangent
theory \owen{? that's not widespread terminology and hence should be explained} to the elliptic moduli problem \owen{also not widespread} described by $\fg_n^\RR$. The
pairing is ``wedge and integrate''. Explicitly, for $\phi,\phi' \in
\Omega^*_c(\RR ; \fg_n)$ and $\psi, \psi' \in \Omega^*_c(\RR ;
\fg_n^\vee[-2])$ we have 
\ben
\<\phi + \psi, \phi' + \psi'\> = \int_\RR \ev_{\fg_n}(\phi \wedge
\psi') + \ev_{\fg_n^\vee} (\psi \wedge \phi') .
\een
This pairing has cohomological degree $-3$ and hence determines a
classical BV theory \owen{no hyphen! BV theory} that we call {\em one-dimensional Chern-Simons
  with target $\hD^n$}. 

\subsection{The $\Vect$ action}

\subsubsection{}

We have just introduced the dg Lie algebra $\fg_n$.
\owen{I find this a weird way to start the subsection. Need to restructure a bit.}

Let us denote the generators of $\fg_n$ by $\{\xi_1,\ldots,\xi_n\}$
and the dual generators by $\{t_1,\ldots,t_n\}$. Thus
\ben
\clie^*(\fg_n) = \hsym(\fg^\vee_n[-1]) = \CC [[  t_1,\ldots, t_n ]].
\een 
There is hence a natural isomorphism
\ben
\rho^{\rm W} : \Vect \to \Der(\clie^*(\fg_n))
\een
sending $f(t_i) \partial_j$ to $f(t_i) \xi_j$. 
\owen{Might help to identify the derivations with something explicit so that these notations make sense.}

We can interpret $\rho^{\rm W}$ as expressing an $\L8$ action of $\Vect$ on $\fg_n$, where
\ben
\ell^W_m : \Vect \otimes \fgn^{\otimes m} \to \fgn
\een
has cohomological degree $1-m$ and $m$ ranges over all non-negative integers.
These maps are simply the ``Taylor components'' of $\rho_W$.
For instance, the vector field $\fX = t_1^{m_1} \cdots t_n^{m_n} \partial_j \in \Vect$ acts by zero for any $m \neq m_1 + \cdots + m_n$, 
and for $m = m_1 + \cdots + m_n$, 
\ben
\ell^W_m\left(\fX, (\xi_1^{\otimes m_1} \otimes \cdots \otimes \xi_n^{\otimes m_n}) \right) 
= \ell_m^\bW \left((t_1^{m_1} \cdots t_n^{m_n} \partial_j) \otimes \xi_1^{\otimes m_1}\otimes \cdots \otimes \xi_n^{\otimes m_n} \right) 
= \xi_j 
\een 
and vanishes on any other basis element $\fg_n^{\otimes m}$.

\subsubsection{}

Let $A$ be a commutative dg algebra. 
We now show how the
dg Lie algebra $A \otimes \fgn$ inherits a natural $\L8$ action of $\Vect$.
Here the sequence of maps is
\ben
\ell^{W,A}_m : \Vect \otimes (A \otimes \fgn)^{\otimes m} \to A \otimes \fgn
\een
with
\[
\ell^{W,A}_m(\fX, (a_1 \otimes x_1)\otimes \cdots \otimes (a_m \otimes x_m)) = \pm (a_1\cdots a_m) \otimes \ell^W_m(\fX,x_1 \otimes \cdots \otimes x_m),
\]
where the sign is determined by Koszul's rule.
Equivalently, we can encode the $\L8$ action in a Lie algebra map
\[
\rho_{W,A}: \Vect \to \clies(A \otimes \fgn, A \otimes \fgn[-1]),
\]
which assembles the $\ell^A_m$ maps into a ``Taylor series.''
If we set $A$ to be $\Omega^{*}(\RR)$, then we obtain an $\L8$ action of $\Vect$ on $\fg_n^\RR$.
A lift of this action to an $\L8$ action of $\Vect$ on $\DD\fg_n^\RR$
is uniquely determined by the requirement that the action preserve the
pairing of degree $-3$. 

\owen{Might be nice to make a small comment about why this construction is relevant (encodes action on a mapping space!}

\subsection{The deformation complex}


\begin{prop}\label{propdef} 
There is a quasi-isomorphism of $\Vect$-modules
\ben
J : \Omega^1_{n,cl}[1] \xto{\simeq} \left(\Def_n\right)^{\RR^\times \times \Aff(\RR)}.
\een
\owen{Is it a map of Lie algebras?}
Thus, taking the invariant part of the $\Vect$-equivariant deformation complex, 
we have a quasi-isomorphism
\ben
J : \clie^*(\Vect ; \hOmega^1_{n,cl}[1]) \xto{\simeq}\left(\Def_n^{\rm W}\right)^{\RR^\times \times \Aff(\RR)}
.
\een 
\end{prop}

\subsubsection{One-forms as local functionals}

We explicitly define the quasi-isomorphism 
\ben
J : \hOmega^1_{n,cl}[1] \to \Def_n
\een
as follows. First, we note that this quasi-isomorphism will land completely in the
$\RR^\times$-invariant piece of the deformation complex 
\ben
\cloc^*(\fg_n^\RR) \subset \Def_n .
\een 
Given $\theta \in \hOmega^1_n= \clie^*(\fg_n ;
  \fg_n^\vee[-1])$ we define
\ben
J_\theta(\gamma) = \sum_k \int_{\RR} \<\theta_k(\gamma^{\tensor k}),
\gamma\>_{\fg}
\een
where $\theta_k$ is the $k$th homogenous component of $\theta$, that
is $\theta_k \in \Sym^k(\fg_n^\vee[-1]) \tensor \fg_n^\vee[-1]$.  

\owen{I think we provided a nicer geometric motivation for this construction in the latest version of the CDO paper.}

\begin{prop} The map $J$ has the following properties: 
\begin{itemize}
\item[(1)] For each $\theta$, $J_\theta$ is a local functional and is invariant with respect
  to the action of $\Aff(\RR)$.
\item[(2)] The assignment $\theta \mapsto J_\theta$ is
  $\Vect$-equivariant. 
\item[(3)] If $\theta$ is closed then $\d_{dR} J_\theta = 0$. 
\end{itemize}
\end{prop} 

This proposition implies that we have a $\Vect$-equivariant cochain map 
\ben
J : \hOmega^1_{n,cl} [1] \to \cloc^*(\fg_n^\RR)^{\Aff(\RR)} =
\left(\Def_n\right)^{\RR^\times \times \Aff(\RR)} .
\een
Note the shift on the left side is due to the fact that $J_\theta$ is
a local functional of degree $-1$.

\subsubsection{Proof of Proposition \ref{propdef}}

To complete the proof of the proposition it remains to show that $J$
is a quasi-isomorphism. This is a cohomological calculation. We first recall the non-scale invariant deformation complex.

\begin{lem}[Corollary 5.2 of \cite{GG1}]
We have a quasi-isomorphism
\[
\hOmega^1_{n,cl} \oplus \hOmega^1_{n,cl}[1] \xrightarrow{\; \sim \;} \left(\Def_n\right)^{\Aff(\RR)} 
\]
\end{lem}

We are left to verify that the map of the preceding lemma restricted to the second summand is $J$.  We also need to show that deformations coming from $\hOmega^1_{n,cl} [0]$ are not $R^\times$-invariant. Both of these claims are explicit computations.  


First, recall that we have a quasi-isomorphism
\[
C^\ast_{red} (\fg_n) \oplus C^\ast_{red} (\fg_n) [1] \simeq \hOmega^1_{n,cl} \oplus \hOmega^1_{n,cl}[1].
\]
Next, let $(\cR^\ast, \delta)$ be the Koszul resolution of the trivial $\RR[\partial/\partial x]$-module $\RR dx$ i.e., let $\epsilon$ be square zero of cohomological degree -1 so $\cR^\ast = \RR [\epsilon, \partial/\partial x] dx$ with 
\[
\delta \epsilon = \frac{\partial}{\partial x} \; \; \text{ and } \; \; \delta \left ( \frac{\partial}{\partial x} \right )^k = \left ( \frac{\partial}{\partial x} \right )^{k+1} 
\]
We therefore need to trace from bottom right to bottom left in the following diagram. (Note that the quasi-isomorphism $PL$ follows from the Poincar\'e Lemma.)

\begin{footnotesize}
\[
\xymatrix{ & \cR^\ast  \otimes_{\RR[\partial/\partial x]}   C^\ast_{red} (\fg_n [[x, dx]]) \ar[1,-1]_\simeq^{\epsilon \sim 0} \ar[1,1]^{\simeq}_{PL} &\\
 \left (\mathrm{Dens}_\RR \right )^{\Aff(\RR)} \otimes_{D(\RR)^{\Aff(\RR)}}  \sO(J(E)_0)/\RR \ar[d]_{\cong} &&\cR^\ast \otimes_{\RR[\partial/\partial x]} C^\ast_{red} (\fg_n) \ar[d]^{\cong} \\
 \left (\sO_{loc} (\sE)/\RR \right )^{\Aff(\RR)} && C^\ast_{red} (\fg_n) \epsilon dx \oplus C^\ast_{red} (\fg_n) dx
}
\]
\end{footnotesize}

It is straightforward (\ryan{too much?}) to see that a closed element $\theta \epsilon dx \in C^\ast_{red} (\fg_n) \epsilon dx$ maps to $J_\theta$ under this correspondence. Also, an element $f \in C^\ast_{red} (\fg_n)dx$ maps to integration of a potential function.  Any non-zero potential function will not be scale invariant.


\ryan{Where do we use this lemma, it seems duplicated below?}

\begin{lem} The first cohomology of $\Vect$ with coefficients in the
  module $\hOmega^1_{n,cl}$ is one dimensional
\ben
{\rm H}^1(\Vect ; \hOmega^1_{n,cl}) \cong \CC
\een
spanned by the Gelfand-Fuks chern class $c^{\rm GF}_1(\hT_n)$. 
\end{lem}

\subsection{BV quantization}

\brian{We should copy a lot of the analysis from Paper 1 and put in
  the next 3 sections with modest notational modification.}
  
 \ryan{Equivariant quantization}

\subsubsection{Pre-quantization}

\subsubsection{Vanishing of the obstruction}


\subsection{Quantum Observables}

\ryan{We just focus on smooth observables...fix notation below...blah...}

Let's start with the classical factorization algebra which assigns to
an interval $I \subset \RR$ the cochains on the dg Lie algebra $\DD
\fg_n^I$. If $I \subset J$ is an embedding of open intervals then it
follows from the Poincar\'{e} Lemma for differential forms on $\RR^n$
that the induced map
\ben
\fg_n^J \xto{\simeq} \fg_n^I
\een
is a homotopy equivalence. It follows that $\Obs^{cl}_n$ is a locally
constant factorization algebra. 

The aim of this section is to prove the following theorem.

\begin{thm}\label{thm:formalobsq} Let $I \subset \RR$ be a non-empty interval.  
\begin{itemize}
\item[(a)] There is a quasi-isomorphism $F_I : \hA_n \xrightarrow{\sim} Obs^q_n (I)$.  
\item[(b)] The quasi-isomorphism extends to a $\Vect$-equivariant quasi-isomorphism.
\end{itemize}
\end{thm}

\begin{cor}
There is an equivalence of factorization algebras $\cF_{\hA_n} \xrightarrow{\sim} Obs^q_n$, from the one-dimensional factorization algebra determined by the associative algebra $\hA_n$ to the quantum observables of the formal theory on the formal $n$-disk.
\end{cor}


\subsubsection{The non-equivariant observables}

This follows from Section \ref{sect:free}...

Let $f_I$ be a smooth bump function of integral 1 with supported contained in the interval $I \subset \RR$.  There is a commutative diagram
\[
\xymatrix{
\hA_n \ar[r]^{F_I} \ar@{=}[d] & Obs^q_n (I) \ar[d]\\ \CC[[t_i , \partial t_i ]] \ar[r]^{f_I} & Obs^{cl}_n (I) \ar@/_1pc/[u]
}
\]
We abuse notation in the diagram above as the bottom map is determined by the bump function $f_I$, more explicitly, it is the assignment
\[
t_i \mapsto f_I dx \otimes t_i \quad \text{ and } \quad \partial t_i \mapsto f_I dx \otimes \partial t_i.
\]

\subsubsection{$\Vect$-equivariant observables}

We now prove part (b) of Theorem \ref{thm:formalobsq}.  Define a map
\[
F_I^\Vect : C^\ast (\Vect ; \hA_n) \to \left ( \sO (\Vect[1]) \otimes Obs^q_n (I) , \widetilde{Q} \right ); \quad F_I^\Vect = \mathrm{Id} \otimes F_I.
\]
Recall from Section \ref{sect:eq_obs}, that the codomain of the map $F^\Vect_I$ is our definition of the $\Vect$-equivariant quantum observables on an interval $I$.

\begin{prop}
The map $F_I^\Vect : C^\ast (\Vect ; \hA_n) \to Obs^{q, \Vect}_n (I)$ is a cochain map.
\end{prop}


\ryan{rewrite the following paragraph}


To show that the identification of $\cA_n$ with $\hA_n$ is
$\Vect$-equivariant we must show that the BV bracket with t. Let $a \in \cA_n$ and pick
a lift 
\ben
\Tilde{a} \in \bObs^q(I) .
\een 
Running tree level RG-flow we obtain a (non-smeared) observable
$\{\Tilde{a}[L]\}$. Since the action of a formal vector field $\fX$ on
$\Tilde{a}[L]$ is through the BV bracket it suffices to compute
$\{I^{\rm W}_\fX [L] , \Tilde{a}[L]\}_L$. Likewise, the action of an element
$\theta \in \hOmega^1_{n,cl}$ is the BV bracket with $J_\theta$.

Since $I_\fX^{\rm W}$ is a local functional we see that the scale $L$
bracket is equal
to $\{I_\fX^{\rm W}, \Tilde{a}\}[L] = W(P_0^L, \{I_\fX^{\rm W},
\Tilde{a}\})$. Likewise for the local functional $J$ and closed
one-forms. So, to compute the action of $\Vect \ltimes \hOmega^1_{n,cl}$ on the
quantum observables it suffices to compute $\{I_\fX^{\rm W} + \hbar
J_\theta , -\}$ acting on observables.

Hence, we are left to prove the following lemma.

\begin{lem} Let $\fX \in \Vect$ and $\theta \in \hOmega^1_{n,cl}$. At the level of cohomology in $\Obs^q_n(I)$ we have
\ben
\left[ \{I_\fX^{\rm W}, a\} \right] = \fX \cdot [a]
\een
and 
\ben
\left[ \{J_\theta, a\} \right] = \theta \cdot [a]
\een
where on the right hand side we mean the natural action of formal
vector fields and closed one-forms on the Weyl algebra $\hA_n$ by derivations. 
\end{lem}
\begin{proof}

Let us first consider the local functional $I^{\rm W}_\fX \in \cloc^*(\DD
\fg_n^\RR)$ associated to a fixed formal vector field $\fX \in \Vect$. It is
of the form $I^{\rm W}_\fX = \int \cL_\fX \d x$ where $\cL_\fX : \DD \fg_n \to
C^\infty(\RR)$ is the Lagrangian density. 

Given any element $\omega \in \Omega^*_c(U)$ supported on an interval
$U$ we obtain an observable in $\Obs^{cl}(U)$ by wedging and
integrating 
\ben
\int \omega \wedge \cL_\fX \in \Obs^{cl}_n (U) . 
\een 
In particular, suppose $f$ is a bump function for the interval
$U$. That is, $\int_U f \d x = 1$. Then, we can consider the
observable 
\ben
\int f \d x \wedge \cL_\fX \in \Obs^{cl}_n(U) .
\een
It is easy to see that the cohomology class of this observable
coincides with the element $\fX \in \cA_n$. \brian{should we write down
  the formulas?}

If we view $\fX$ as an element of the formal Weyl algebra then the
action $\fX \cdot a$ is equal to the commutator $\frac{1}{\hbar} [\fX,
a]$. We wish to show that the image of $\{I_\fX^{\rm W}, \tilde{a}\}$ in
cohomology is equal to this commutator, which we proceed to compute.

Suppose $U_{-1}$, $U_0$, $U_1$ are the intervals of length $1/2$
centered at $-1$, $0$, and $1$ respectively. Fix a bump function $f$
for the interval $U_0$ and let
\ben
(\tau_{-1} f)(x) = f(x+1) \;\; , \;\; (\tau_1 f)(x) = f(x-1) .
\een 
Define the function 
\ben
h (x) = \int_{s = 0}^{s = x} \left((\tau_{1} f) - (\tau_{-1} f) \right) \d
s .
\een

Fix a closed element $a \in \Obs_n^q(U_0)$ and consider the observable
$\int h \wedge \cL_\fX \in \Obs^q_n(\RR)$. The following is a direct calculation 
\bestar
(\d_{dR} + \hbar \Delta) \left( \left(\int h \wedge \cL_\fX\right) \cdot
  a \right) & = & 
\left(\int (\d_{dR} h) \wedge \cL_\fX \right) \cdot a 
+ \hbar \Delta \left( \left(\int h \wedge \cL_\fX\right) \cdot a \right) \\ & = & \left(\int (f_{1} \d x) \wedge \cL_\fX - \int (f_{-1} \d x)\wedge
    \cL_\fX \right) \cdot a - \{I_\fX, a\} .
\eestar 
We have used the fact that $a$
is a closed quantum observable by assumption so $(\d_{dR} + \hbar
\Delta) a = 0$. We have also used the fact that $h$ is identically
$-1$ on the interval $U_0$ so that $\int h \cL_\fX = I_\fX$ on $U_0$. 

We conclude that in the complex $\Obs^q_n(\RR)$ the following relation holds
\ben
\hbar \{I^{\rm W}_\fX , a\} = I_\fX^{\rm W} ((\tau_{1} f) \d t - (\tau_{-1} f) \d t) \cdot
a + \{{\rm exact\;terms}\} .
\een
In cohomology, the right hand side becomes $[\fX, [a] ]$ as desired,
where $[a]$ is the cohomology class of $a$. 
\end{proof}






The proof of Theorem \ref{thm:formalobsq} is completed by the following lemma.

\begin{lem}
The map $F_I^{\Vect}$ is a quasi-isomorphism.
\end{lem}