\documentclass[11pt]{amsart}

\linespread{1.25}

\usepackage{macros}

\def\brian{\textcolor{blue}{BW: }\textcolor{blue}}
\def\owen{\textcolor{red}{OG: }\textcolor{red}}
\def\ryan{\textcolor{green}{RG: }\textcolor{green}}

\title{MRC Letter of Intent}

\begin{document}
\maketitle

\section*{Mathematical Topic}

We propose a Mathematical Research Community (MRC) in ``Derived Geometry: Interactions with Quantum Field Theory.'' Note MSRI program is Spring 2019, could leverage to increase number of young researchers (particularly from underrepresented groups) in the field. \brian{there may be some standardized format for explaining the topic, but I don't understand that sentence. Should we say something like
"This MSRI program in Spring 2019 could leverage to increase number of young researchers (particularly from underrepresented groups) in the field."?}

\section*{Key Senior Personnel} Would be nice to have some diversity among the 3--6. \ryan{Probably should remove myself from the following list!}

\begin{enumerate}
\item \label{Sarah} Sarah Scherotzke
\item \label{Damien} Damien Calaque
\item \label{DavidJ} David Jordan
\item \label{DavidA} David Ayala
\item \label{Ryan} Ryan Grady
\item \label{DavidBZ} David Ben-Zvi
\item \label{PavelS} Pavel Safronov
\item \label{Matt} Matt Szczesny 
\end{enumerate}

\section*{Problems on which to Concentrate}

\begin{itemize}
\item[(a)] The categorified Chern character via QFT.  There is a refinement of the result of Ben-Zvi, Francis, and Nadler by Schertotzke and others \cite{HSS}.  Their result should have a natural interpretation/extension via topological quantum field theory. This would be the specialty of (\ref{Sarah}) and (\ref{DavidBZ}).
\brian{can't really comment on this, as I haven't looked at it at all. Sounds really interesting and has intersection with the senior personnel.}
\item[(b)] Supersymmetry for derived stacks and derived spin structures. 
Develop Costello and Gwilliam's approach to perturbative field theory for supersymmetric space-times. For instance, in dimension two this should relate to the theory of superconformal vertex algebras. 
This is related to the Stolz-Teichner program for elliptic cohomology. 
This could be led by (\ref{Damien}), (\ref{Ryan}), and (\ref{Matt}). \brian{Can reference DavidBZ and Matt's work on superconformal vertex algebras}. 
\item[(c)] An extension of AKSZ to shifted Poisson structures and degenerate field theories: the CME and BRST complex. 
Subtopics include applications to generalized geometry and Courant algebroids. 
This could be led by (\ref{Damien}) or (\ref{Ryan}) with (\ref{PavelS}).
\item[(d)] Diffeomorphism groups or Bruhat decompositons via factorization homology. 
This could be led by (\ref{DavidA}).
\item[(e)] Topological orders/phases via factorization homology. 
This would aim to clarify for a mathematical community the recent computations/assertions of Liang Kong and Xiao-Gang Wen. 
This could be led by (\ref{DavidJ}) possibly with the help of (\ref{DavidBZ}) or (\ref{DavidA}).

\end{itemize}

I started to make a bib, but I realize that may not be what you wanted. 

\bibliographystyle{plain}

\bibliography{LOI}


\end{document}