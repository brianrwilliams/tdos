\documentclass[11pt]{amsart}

\linespread{1.25}

\usepackage{macros}

\def\brian{\textcolor{blue}{BW: }\textcolor{blue}}
\def\owen{\textcolor{red}{OG: }\textcolor{red}}
\def\ryan{\textcolor{green}{RG: }\textcolor{green}}

\title{MRC Letter of Intent}

\begin{document}
\maketitle

\section*{Mathematical Topic}

We propose a Mathematical Research Community (MRC) in ``Derived Geometry: Interactions with Quantum Field Theory.'' Note MSRI program is spring 2019, could leverage to increase number of young researchers (particularly from underrepresented groups) in the field.

\section*{Key Senior Personnel} Would be nice to have some diversity among the 3--6. \ryan{Probably should remove myself from the following list!}

\begin{enumerate}
\item Sarah Scherotzke
\item Damien Calaque
\item David Jordan
\item David Ayala
\item Ryan Grady
\item David Ben-Zvi
\item Pavel Safronov
\end{enumerate}

\section*{Problems on which to Concentrate}

\begin{itemize}
\item[(a)] The categorified Chern character via QFT.  There is a refinement of the result of Ben-Zvi, Francis, and Nadler by Schertotzke and others \href{https://arxiv.org/abs/1511.03589}{arXiv:1511.03589}.  Their result should have a natural interpretation/extension via topological quantum field theory. This would be the specialty of (1) and (6).
\item[(b)] Supersymmetry for derived stacks and derived spin structures. This could be led by (2) and (5).
\item[(c)] An extension of AKSZ to shifted Poisson structures and degenerate field theories: the CME and BRST complex. This could be led by (2) or (5) with (7).
\item[(d)] Diffeomorphism groups or Bruhat decompositons via factorization homology. This could be led by (4).
\item[(e)] Topological orders/phases via factorization homology. This would aim to clarify for a mathematical community the recent computations/assertions of Liang Kong and Xiao-Gang Wen. This could be led by (3) possibly with the help of (6) or (4).

\end{itemize}




\section*{Literature}



\end{document}