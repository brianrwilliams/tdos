\section{The Formal Theory}


\owen{This section should also have an introduction that indicates how it is parallel to the purely algebraic construction just described.}

We construct one-dimensional Chern-Simons \owen{theory} with target the formal
disk. We will use this theory to study how diffeomorphisms on the target act
on the theory, which will be encoded by the structure of
$\Vect$-equivariance. 
\owen{This second sentence is befuddling.}

We have already seen \owen{?} that the formal disk is encoded by the dg Lie
algebra $\fg_n = \CC^n[-1]$. Consider the dg Lie algebra 
\ben
\fg_n^\RR := \Omega^*(\RR ; \fg_n) .
\een  
This dg Lie algebra is abelian and the differential is the de Rham
differential $\d_{dR}$. This dg Lie algebra encodes deformations of the
constant map to 0 to nearby smooth functions. 

The dg Lie algebra that encodes one-dimensional Chern-Simons with
target $\hD^n$ is defined to be the ``double''
\ben
\DD \fg_n^\RR = \Omega^*(\RR ; \fg_n \oplus \fg_n^\vee[-2]) .
\een 
The Lie bracket is still trivial, and the differential is
$\d_{dR}$. This dg Lie algebra encodes the cotangent
theory \owen{? that's not widespread terminology and hence should be explained} to the elliptic moduli problem \owen{also not widespread} described by $\fg_n^\RR$. The
pairing is ``wedge and integrate''. Explicitly, for $\phi,\phi' \in
\Omega^*_c(\RR ; \fg_n)$ and $\psi, \psi' \in \Omega^*_c(\RR ;
\fg_n^\vee[-2])$ we have 
\ben
\<\phi + \psi, \phi' + \psi'\> = \int_\RR \ev_{\fg_n}(\phi \wedge
\psi') + \ev_{\fg_n^\vee} (\psi \wedge \phi') .
\een
This pairing has cohomological degree $-3$ and hence determines a
classical BV theory \owen{no hyphen! BV theory} that we call {\em one-dimensional Chern-Simons
  with target $\hD^n$}. 

\subsection{The $\Vect$ action}

\subsubsection{}

We have just introduced the dg Lie algebra $\fg_n$.
\owen{I find this a weird way to start the subsection. Need to restructure a bit.}

Let us denote the generators of $\fg_n$ by $\{\xi_1,\ldots,\xi_n\}$
and the dual generators by $\{t_1,\ldots,t_n\}$. Thus
\ben
\clie^*(\fg_n) = \hsym(\fg^\vee_n[-1]) = \CC [[  t_1,\ldots, t_n ]].
\een 
There is hence a natural isomorphism
\ben
\rho^{\rm W} : \Vect \to \Der(\clie^*(\fg_n))
\een
sending $f(t_i) \partial_j$ to $f(t_i) \xi_j$. 
\owen{Might help to identify the derivations with something explicit so that these notations make sense.}

We can interpret $\rho^{\rm W}$ as expressing an $\L8$ action of $\Vect$ on $\fg_n$, where
\ben
\ell^W_m : \Vect \otimes \fgn^{\otimes m} \to \fgn
\een
has cohomological degree $1-m$ and $m$ ranges over all non-negative integers.
These maps are simply the ``Taylor components'' of $\rho_W$.
For instance, the vector field $\fX = t_1^{m_1} \cdots t_n^{m_n} \partial_j \in \Vect$ acts by zero for any $m \neq m_1 + \cdots + m_n$, 
and for $m = m_1 + \cdots + m_n$, 
\ben
\ell^W_m\left(\fX, (\xi_1^{\otimes m_1} \otimes \cdots \otimes \xi_n^{\otimes m_n}) \right) 
= \ell_m^\bW \left((t_1^{m_1} \cdots t_n^{m_n} \partial_j) \otimes \xi_1^{\otimes m_1}\otimes \cdots \otimes \xi_n^{\otimes m_n} \right) 
= \xi_j 
\een 
and vanishes on any other basis element $\fg_n^{\otimes m}$.

\subsubsection{}

Let $A$ be a commutative dg algebra. 
We now show how the
dg Lie algebra $A \otimes \fgn$ inherits a natural $\L8$ action of $\Vect$.
Here the sequence of maps is
\ben
\ell^{W,A}_m : \Vect \otimes (A \otimes \fgn)^{\otimes m} \to A \otimes \fgn
\een
with
\[
\ell^{W,A}_m(\fX, (a_1 \otimes x_1)\otimes \cdots \otimes (a_m \otimes x_m)) = \pm (a_1\cdots a_m) \otimes \ell^W_m(\fX,x_1 \otimes \cdots \otimes x_m),
\]
where the sign is determined by Koszul's rule.
Equivalently, we can encode the $\L8$ action in a Lie algebra map
\[
\rho_{W,A}: \Vect \to \clies(A \otimes \fgn, A \otimes \fgn[-1]),
\]
which assembles the $\ell^A_m$ maps into a ``Taylor series.''
If we set $A$ to be $\Omega^{*}(\RR)$, then we obtain an $\L8$ action of $\Vect$ on $\fg_n^\RR$.
A lift of this action to an $\L8$ action of $\Vect$ on $\DD\fg_n^\RR$
is uniquely determined by the requirement that the action preserve the
pairing of degree $-3$. 

\owen{Might be nice to make a small comment about why this construction is relevant (encodes action on a mapping space!}

\subsection{The deformation complex}


\begin{prop}\label{propdef} 
There is a quasi-isomorphism of $\Vect$-modules
\ben
J : \Omega^1_{n,cl}[1] \xto{\simeq} \left(\Def_n\right)^{\RR^\times \times \Aff(\RR)}.
\een
\owen{Is it a map of Lie algebras?}
Thus, taking the invariant part of the $\Vect$-equivariant deformation complex, 
we have a quasi-isomorphism
\ben
J : \clie^*(\Vect ; \hOmega^1_{n,cl}[1]) \xto{\simeq}\left(\Def_n^{\rm W}\right)^{\RR^\times \times \Aff(\RR)}
.
\een 
\end{prop}

\subsubsection{One-forms as local functionals}

We explicitly define the quasi-isomorphism 
\ben
J : \hOmega^1_{n,cl}[1] \to \Def_n
\een
as follows. First, we note that this quasi-isomorphism will land completely in the
$\RR^\times$-invariant piece of the deformation complex 
\ben
\cloc^*(\fg_n^\RR) \subset \Def_n .
\een 
Given $\theta \in \hOmega^1_n= \clie^*(\fg_n ;
  \fg_n^\vee[-1])$ we define
\ben
J_\theta(\gamma) = \sum_k \int_{\RR} \<\theta_k(\gamma^{\tensor k}),
\gamma\>_{\fg}
\een
where $\theta_k$ is the $k$th homogenous component of $\theta$, that
is $\theta_k \in \Sym^k(\fg_n^\vee[-1]) \tensor \fg_n^\vee[-1]$.  

\owen{I think we provided a nicer geometric motivation for this construction in the latest version of the CDO paper.}

\begin{prop} The map $J$ has the following properties: 
\begin{itemize}
\item[(1)] For each $\theta$, $J_\theta$ is a local functional and is invariant with respect
  to the action of $\Aff(\RR)$.
\item[(2)] The assignment $\theta \mapsto J_\theta$ is
  $\Vect$-equivariant. 
\item[(3)] If $\theta$ is closed then $\d_{dR} J_\theta = 0$. 
\end{itemize}
\end{prop} 

This proposition implies that we have a $\Vect$-equivariant cochain map 
\ben
J : \hOmega^1_{n,cl} [1] \to \cloc^*(\fg_n^\RR)^{\Aff(\RR)} =
\left(\Def_n\right)^{\RR^\times \times \Aff(\RR)} .
\een
Note the shift on the left side is due to the fact that $J_\theta$ is
a local functional of degree $-1$.

\subsubsection{Proof of Proposition \ref{propdef}}

To complete the proof of the proposition it remains to show that $J$
is a quasi-isomorphism. This is a cohomological calculation. 

\brian{Insert your guys calculation here, adjusted to the formal disk.}



\begin{lem} The first cohomology of $\Vect$ with coefficients in the
  module $\hOmega^1_{n,cl}$ is one dimensional
\ben
{\rm H}^1(\Vect ; \hOmega^1_{n,cl}) \cong \CC
\een
spanned by the Gelfand-Fuks chern class $c^{\rm GF}_1(\hT_n)$. 
\end{lem}

\subsection{BV quantization}

\brian{We should copy a lot of the analysis from Paper 1 and put in
  the next 3 sections with modest notational modification.}
  
 \ryan{Equivariant quantization}

\subsubsection{Pre-quantization}

\subsubsection{Vanishing of the obstruction}

\subsection{The ``extended'' theory}

\owen{I think it would be good to explain why we want to do this and what it should accomplish, or at least point to some such discussion elsewhere in the paper. Otherwise it's a bit terse.}

We will now extend our $\Vect$-equivariant theory by the module of
deformations (respecting certain symmetries) that we computed
above. Thus, we will consider the extension of the Lie algebra $\Vect$
by the module $\hOmega^1_{n,cl}$ given by $\Vect \ltimes
\hOmega^1_{n,cl}$. Just as above, we show that there is a quantization
of this classical equivariant theory. This theory will admit a useful and
interesting class of deformations when we descend
over arbitrary target manifolds. 

The extended classical theory is simple to describe. The
$\Vect$-equivariant theory above was encoded by a map of Lie
algebras
\ben
I^{\rm W} : \Vect \to \cloc^*(\DD \fg_n^\RR)[-1] .
\een 
Our extended classical theory is encoded by the restriction of this
equivariant theory along the map of Lie algebras $p : \Vect \ltimes
\hOmega^1_{n,cl} \to \Vect$. It is immediate that this defines a $\Vect
\ltimes \hOmega^1_{n,cl}$-equivariant classical field theory. 


Denote by $\Tilde{\Def}^{\rm W}_n$ the extended equivariant
deformation complex. We can realize it as the tensor product
\ben
\Tilde{\Def}^{\rm W}_n = \clie^*(\Vect \ltimes \hOmega^1_{n,cl} ;
\Def_n) \cong \clie^*(\Vect \ltimes \hOmega^1_n)
\tensor_{\clie^*(\Vect)} \Def_n^{\rm W} .
\een
Thus, the quasi-isomorphism $J$ extends to a quasi-isomorphism
\be\label{extdef}
\Tilde{J} : \clie^*(\Vect \ltimes \hOmega^1_n ; \hOmega^1_n)
\xto{\simeq} \left(\Tilde{\Def}_n^{\rm W}\right)^{\RR^\times \times
  \Aff(\RR)} .
\ee
Here, $\hOmega^1_{n,cl}$ denotes the restriction of the $\Vect$-module
along the map $p : \Vect \ltimes \hOmega^1_{n,cl} \to \Vect$. 

There is a natural cocycle in $\clie^1(\Vect \ltimes \hOmega^1_{n,cl} ;
\hOmega^1_{n,cl})$ given by the identity $\id_{\Omega^1}$ on one-forms. 

\begin{lem} The space ${\rm H}^1(\Vect \ltimes \hOmega^1_{n,cl} ,
  \GL_n ; \hOmega^1_{n,cl})$ is two dimensional spanned by $c_1^{\rm
    GF}(\hT_n)$ and $\id_{\Omega^1}$. 
\end{lem}

Thus, the cocycle $\id_{\Omega^1}$ determines an additional cohomologically non-trivial deformation in the extended case. 

\subsubsection{Quantization} 

We obtain a quantization for this classical theory in the same way as
in the non-extended case. In the $\Vect$-equivariant theory we showed
that we have a quantization given by the collection of functionals
$\{I^{\rm W}  [L]\} \subset \clie^\# (\Vect ; \clie^\#(\DD \fg_n^\RR))
[[ \hbar ]]$. We can restrict these functionals along the morphism
$p : \Vect \ltimes \hOmega^1_{n,cl}$ to obtain functionals
\ben
\Tilde{I}^{\rm W}[L] := p^*(I^{\rm W}[L]) \in \clie^\# (\Vect 
\ltimes \hOmega^1_{n,cl} ; \clie^\#(\DD \fg_n^\RR) [[ \hbar ]] .
\een 
The same argument as above shows that this pre-quantization is
actually a quantization; that is, there is no obstruction. 

\begin{prop} The functionals $\{\Tilde{I}^{\rm W}[L]\}$ define a
  quantization for the $\Vect \ltimes \hOmega^1_{n,cl}$-equivariant
  theory.
\end{prop}

\subsubsection{A deformation}

We are interested in a deformation of this theory given by the
additional cocycle present in the extended case. The map $J :
\hOmega^1_{n,cl} \to \Def_n$ from \owen{add cross ref} extends to a map
\ben
J : \Vect \ltimes \hOmega^1_{n,cl} \to \Def_n
\een
that sends $(X,\theta) \mapsto J_\theta$. Hence we can realize $J$
as a cocycle in $\Def_n^{\rm W}$. Tautologically, this cocycle corresponds to
the identity cocycle $\id_{\Omega^1} \in \clie^*(\Vect \ltimes
\hOmega^1_{n,cl} ; \hOmega^1_{n,cl})$ under the quasi-isomorphism~(\ref{extdef}). 

We then obtain a deformation of the quantum theory as follows. For $L > 0$
define the functional 
\ben
J[L] := \lim_{\epsilon \to 0} \sum_{\substack{\Gamma \in
    \text{\rm Trees}\\ v \in V(\Gamma)}} W_{\Gamma, v}(P_{\epsilon <
  L}, I, J)
\een
where the sum is over graphs and distinguished vertices. The notation $W_{\Gamma, v}(P_{\epsilon <
  L}, I^{\rm W}, J)$ means that we compute the weight of the graph $\Gamma$
with $J$ placed at the vertex $v$ and $I^{\rm W}$ placed at all other
vertices. \owen{Add picture!}

\begin{prop} \brian{ref any of Si's papers here, he always proves this
    fact}For each $L > 0$ the functional $I^{\rm W}[L] + \hbar J[L]$
  satisfies the scale $L$ quantum master equation
\ben
(\d_{dR} + \d_{\Vect \ltimes \hOmega^1_{n,cl}}) (I^{\rm W}[L] + \hbar
J[L]) + \frac{1}{2} \{I^{\rm W}[L] + \hbar J[L], I^{\rm W}[L] + \hbar
J[L]\}_L + \hbar \Delta_L (I^{\rm W}[L] + \hbar J[L]) = 0 .
\een
Thus, the family of functionals $\{\Tilde{I}^{\rm W}[L] + \hbar
  J[L]\}_{L > 0}$ defines a quantization of the $\Vect \ltimes
  \hOmega^1_{n,cl}$-equivariant theory. 
\end{prop}