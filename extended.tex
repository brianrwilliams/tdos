\section{A Deformation: Twisted Differential Operators}






We will now extend our $\Vect$-equivariant theory by the module of
deformations (respecting certain symmetries) that we computed
above. Thus, we will consider the extension of the Lie algebra $\Vect$
by the module $\hOmega^1_{n,cl}$ given by $\Vect \ltimes
\hOmega^1_{n,cl}$. Just as above, we show that there is a quantization
of this classical equivariant theory. This theory will admit a useful and
interesting class of deformations when we descend
over arbitrary target manifolds. 

The extended classical theory is simple to describe. The
$\Vect$-equivariant theory above was encoded by a map of Lie
algebras
\ben
I^{\rm W} : \Vect \to \cloc^*(\DD \fg_n^\RR)[-1] .
\een 
Our extended classical theory is encoded by the restriction of this
equivariant theory along the map of Lie algebras $p : \Vect \ltimes
\hOmega^1_{n,cl} \to \Vect$. It is immediate that this defines a $\Vect
\ltimes \hOmega^1_{n,cl}$-equivariant classical field theory. 


Denote by $\Tilde{\Def}^{\rm W}_n$ the extended equivariant
deformation complex. We can realize it as the tensor product
\ben
\Tilde{\Def}^{\rm W}_n = \clie^*(\Vect \ltimes \hOmega^1_{n,cl} ;
\Def_n) \cong \clie^*(\Vect \ltimes \hOmega^1_n)
\tensor_{\clie^*(\Vect)} \Def_n^{\rm W} .
\een
Thus, the quasi-isomorphism $J$ extends to a quasi-isomorphism
\be\label{extdef}
\Tilde{J} : \clie^*(\Vect \ltimes \hOmega^1_n ; \hOmega^1_n)
\xto{\simeq} \left(\Tilde{\Def}_n^{\rm W}\right)^{\RR^\times \times
  \Aff(\RR)} .
\ee
Here, $\hOmega^1_{n,cl}$ denotes the restriction of the $\Vect$-module
along the map $p : \Vect \ltimes \hOmega^1_{n,cl} \to \Vect$. 

There is a natural cocycle in $\clie^1(\Vect \ltimes \hOmega^1_{n,cl} ;
\hOmega^1_{n,cl})$ given by the identity $\id_{\Omega^1}$ on one-forms. 

\begin{lem} The space ${\rm H}^1(\Vect \ltimes \hOmega^1_{n,cl} ,
  \GL_n ; \hOmega^1_{n,cl})$ is two dimensional spanned by $c_1^{\rm
    GF}(\hT_n)$ and $\id_{\Omega^1}$. 
\end{lem}

Thus, the cocycle $\id_{\Omega^1}$ determines an additional cohomologically non-trivial deformation in the extended case. 

\subsubsection{Quantization} 

We obtain a quantization for this classical theory in the same way as
in the non-extended case. In the $\Vect$-equivariant theory we showed
that we have a quantization given by the collection of functionals
$\{I^{\rm W}  [L]\} \subset \clie^\# (\Vect ; \clie^\#(\DD \fg_n^\RR))
[[ \hbar ]]$. We can restrict these functionals along the morphism
$p : \Vect \ltimes \hOmega^1_{n,cl}$ to obtain functionals
\ben
\Tilde{I}^{\rm W}[L] := p^*(I^{\rm W}[L]) \in \clie^\# (\Vect 
\ltimes \hOmega^1_{n,cl} ; \clie^\#(\DD \fg_n^\RR) [[ \hbar ]] .
\een 
The same argument as above shows that this pre-quantization is
actually a quantization; that is, there is no obstruction. 

\begin{prop} The functionals $\{\Tilde{I}^{\rm W}[L]\}$ define a
  quantization for the $\Vect \ltimes \hOmega^1_{n,cl}$-equivariant
  theory.
\end{prop}

\subsubsection{A deformation}

We are interested in a deformation of this theory given by the
additional cocycle present in the extended case. The map $J :
\hOmega^1_{n,cl} \to \Def_n$ from \owen{add cross ref} extends to a map
\ben
J : \Vect \ltimes \hOmega^1_{n,cl} \to \Def_n
\een
that sends $(X,\theta) \mapsto J_\theta$. Hence we can realize $J$
as a cocycle in $\Def_n^{\rm W}$. Tautologically, this cocycle corresponds to
the identity cocycle $\id_{\Omega^1} \in \clie^*(\Vect \ltimes
\hOmega^1_{n,cl} ; \hOmega^1_{n,cl})$ under the quasi-isomorphism~(\ref{extdef}). 

We then obtain a deformation of the quantum theory as follows. For $L > 0$
define the functional 
\ben
J[L] := \lim_{\epsilon \to 0} \sum_{\substack{\Gamma \in
    \text{\rm Trees}\\ v \in V(\Gamma)}} W_{\Gamma, v}(P_{\epsilon <
  L}, I, J)
\een
where the sum is over graphs and distinguished vertices. The notation $W_{\Gamma, v}(P_{\epsilon <
  L}, I^{\rm W}, J)$ means that we compute the weight of the graph $\Gamma$
with $J$ placed at the vertex $v$ and $I^{\rm W}$ placed at all other
vertices. \owen{Add picture!}

\begin{prop} \brian{ref any of Si's papers here, he always proves this
    fact}For each $L > 0$ the functional $I^{\rm W}[L] + \hbar J[L]$
  satisfies the scale $L$ quantum master equation
\ben
(\d_{dR} + \d_{\Vect \ltimes \hOmega^1_{n,cl}}) (I^{\rm W}[L] + \hbar
J[L]) + \frac{1}{2} \{I^{\rm W}[L] + \hbar J[L], I^{\rm W}[L] + \hbar
J[L]\}_L + \hbar \Delta_L (I^{\rm W}[L] + \hbar J[L]) = 0 .
\een
Thus, the family of functionals $\{\Tilde{I}^{\rm W}[L] + \hbar
  J[L]\}_{L > 0}$ defines a quantization of the $\Vect \ltimes
  \hOmega^1_{n,cl}$-equivariant theory. 
\end{prop}





\ryan{This is from an old draft...}



\subsection{Extended descent}

We wish to extend the above construction of Gelfand-Kazhdan descent by
an element $\alpha \in {\rm H}^1(X ; \Omega^1_{cl,X})$.
\owen{This sentence is a bit cryptic. We don't extend the construction by a 2-form but rather work with an HC extension.}
This will give us a formal construction of the sheaf of twisted
differential operators for any $\alpha$ as above.
\owen{"Formal" here is needlessly confusing. We mean that we can construct TDOs by a version of formal geometry.}

Let $\hOmega^1_{n,cl}$ denote the closed one-forms on $\hD^n$.
\owen{We should say somewhere (and at least remind here) whether we mean the cochain complex obtained by truncating the de Rham complex or the actual closed one-forms.} 
The $(\Vect, \GL_n)$-structure comes from Lie derivative by vector fields on forms and linear
changes of frame. From above, we know that for any
choice of a Gelfand-Kazhdan structure $\sigma$ on $X$, we have an
isomorphism of sheaves $\sdesc(\sigma, \hOmega^1_{n,cl}) \cong \Omega^1_{X,cl}$.

Consider the extension  
\ben
0 \to \hOmega^1_{n,cl} \to \Vect \ltimes \hOmega^1_{n,cl} \to \Vect \to
0 
\een
where 
\ben
[X,\omega] = L_X \omega 
\een
for $X$ a vector field \owen{We've already used $X$ for the manifold} and $\omega$ lives in $\hOmega^1_{n,cl}$.
Since $\omega$ is closed this
can be written as $L_X \omega~=~\d(\iota_X \omega)$.

The Lie algebra $\Vect$ is a sub-Lie algebra of $\Vect \ltimes
\hOmega^1_{n,cl}$. The Lie group $\GL_n$ acts on $\Vect$ by linear
changes of frame and this action extends to an action on $\Vect \ltimes
\hOmega^{1}_{n,cl}$. We summarize the situation as follows.

\begin{lem} 
$(\Vect \ltimes \hOmega^1_{n,cl}, \GL_n)$ forms a Harish-Chandra pair. 
\end{lem}
 
There is a natural quotient map of HC pairs
\ben
(\Vect \ltimes \hOmega^1_{n,cl}, \GL_n) \to (\Vect, \GL_n) .
\een 
We wish to understand lifts of the $(\Vect, \GL_n)$-bundle $(\Fr_X,
\omega^\sigma)$ along this quotient map.

For a fixed GK-structure $\sigma$, a lift is the structure of a flat $(\Vect \ltimes
\hOmega^1_{n,cl}, \GL_n)$-principal bundle on a pair $(\Fr_X, \omega)$
such that under the map
\ben
\Omega^1(\Fr_X ; \Vect \ltimes \hOmega^1_{n,cl}) \to \Omega^1(\Fr_X ;
\Vect),
\een
the one-form $\omega$ is sent to $\omega^\sigma$. 

\begin{prop} 
Fix a GK-structure $\sigma$ on $X$. 
There is a natural bijection between the following two sets:
\begin{itemize}
\item ${\rm H}^1(X, \hOmega^1_{X,cl}) \cong {\rm H}^2_{\dR}(X)$ and
\item the set of lifts of the flat $(\Vect, \GL_n)$-principal bundle $(\Fr_X, \omega^\sigma)$ to a flat $(\Vect \ltimes \hOmega^1_{n,cl}, \GL_n)$-principal bundle, up to isomorphism.
\end{itemize}
\end{prop}

\begin{proof}
Consider the sheaf of closed one-forms $\Omega^1_{X, cl}$ and its
associated sheaf of $\infty$-jets $\sJ(\Omega^1_{X,cl})$. 
By definition of the coordinate bundle, 
the pull-back of $\sJ(\Omega^1_{X,cl})$ along 
the projection $\pi^{coor} : X^{coor} \to X$ returns the
trivial sheaf of formal closed one-forms
\ben
(\pi^{coor})^*\left(\sJ(\Omega^1_{X,cl})\right) \cong \ul{\hOmega^1_{n,cl}}.
\een
Any $\alpha \in {\rm H}^1(X ; \Omega^1_{X,cl})$ defines an element in ${\rm H}^1(X ; \sJ(\Omega^1_{X,cl}))$ 
and hence determines an element
\ben
(\pi^{coor})^* \alpha \in {\rm H}^1(X^{coor} ; \ul{\hOmega^1_{n,cl}}) 
\een 
via pull-back to the coordinate bundle.
In turn, this element classifies a $\hOmega^1_{n,cl}$-torsor over $X^{coor}$ that we denote $X^{coor}_\alpha$. 
Being a torsor for a constant sheaf of abelian groups, that are contractible as topological spaces, there exists sections $\sigma_{\alpha} : X^{coor} \to X^{coor}_\alpha$. 
\owen{This last sentence is a little hard to parse. I think "sheaf of vector spaces" would be clearer than "abelian groups." Moreover, don't we have a cochain complex here? Then contractibility sounds weirder. I think we can phrase things so it's a little less distracting and the reader can see that sections obviously exist.}

Recall how one constructs the principal $(\Vect, \GL_n)$-bundle structure on $\Fr_X$ from a GK-structure $\sigma$. 
We have a sequence of bundle maps 
\[
X^{coor} \to \Fr_X \to X,
\]
and $\sigma$ determines a  splitting of the first map.
The flat connection one-form on $\Fr_X$ is the pull-back of the Grothendieck connection $\sigma^*
\omega^{coor}$. This Grothendieck connection is uniquely determined by the transitive
action of $\Vect$ on $X^{coor}$. 
Now we examine how this process would apply to~$X^{coor}_\alpha$.

While $\Vect$ does not act transitively on $X^{coor}_\alpha$, 
the extension $\Vect \ltimes \hOmega^1_{n,cl}$ does act. 
Indeed, we have a pull-back diagram of Lie algebras
\ben
\xymatrix{
\Vect \ltimes \hOmega^1_{n,cl} \ar[d] \ar[r]^-{\theta^{coor}_{\alpha}} & \cX(X^{coor}_\alpha) \ar[d] \\
\Vect \ar[r]^-{\theta^{coor}} & \cX(X^{coor}) .
}
\een
The map $\theta^{coor}$ induces an isomorphism on tangent spaces
$\theta^{coor} : \Vect \xto{\cong} T_{\varphi} X^{coor}$ and hence so
does $\theta^{coor}_{\alpha}$. 
The map $(\theta_\alpha^{coor})^{-1}$ is represented by an element
\ben
\omega^{coor}_\alpha \in \Omega^1(X^{coor} , \Vect \ltimes \hOmega^1_{n,cl})
\een
satisfying the Maurer-Cartan equation. 

Conversely, if we are given a $(\Vect \ltimes \hOmega^1_{n,cl})$-structure $(\Fr_X, \omega)$, 
we can consider the characteristic map 
\ben
\ch : \clie^*(\Vect \ltimes \hOmega^1_{n,cl} , \GL_n ;
\hOmega^1_{n,cl}) \to \bdesc((\Fr_X, \omega), \hOmega^1_{n,cl}) .
\een 
For any tensor bundle $\cV$, we have a quasi-isomorphism 
\ben
\bdesc((\Fr_X, \omega^\sigma), \cV) \simeq \check{\rm C}^*(X ;
\sdesc((\Fr_X, \omega^\sigma), \cV),
\een
since $\bdesc((\Fr_X, \omega^\sigma), \cV)$ is an acyclic resolution
for $\sdesc((\Fr_X, \omega^\sigma), \cV)$. Thus, we have 
\ben
\bdesc((\Fr_X, \omega), \hOmega^1_{n,cl}) \simeq \check{\rm C}^*(X ;
\sdesc((\Fr_X, \omega), \hOmega^1_{n,cl})) = \check{\rm C}^*(X ;
\Omega^1_{X, cl}) .
\een 

\owen{What does the check mean? Do we mean \v{C}ech cohomology? We should figure out what notation we want to use. I thought here we would probably just mean levelwise global sections, since we're talking about a truncated de Rham complex.}

Consider the cocycle $\id_{\Omega^1_{n,cl}} \in \clie^1(\Vect \ltimes
\hOmega^1_{n,cl} ; \hOmega^1_{n,cl})$. In cohomology, the image of
this element under the characteristic map is an element
\ben
\ch([\id_{\Omega^1_{n,cl}}]) \in {\rm H}^1(X ; \Omega^1_{X, cl}) .
\een 
These two constructions are inverse to each other, as can be shown by direct computation. 
\end{proof}

\subsubsection{Extended Harish-Chandra modules}

In this section we define the analog of the category of modules $\VB_{(\Vect,
  \GL_n)}$ for the pair $(\Vect \rtimes \hOmega^1_{n,cl}, \GL_n)$. 

First, consider the category of {\it all} Harish-Chandra modules for the pair $(\Vect \ltimes \hOmega^1_{n,cl}, \GL_n)$. 
There is a subcategory of this consisting of those modules that are filtered with
respect to the obvious two-step filtration on $\Vect \ltimes
\hOmega^{1}_{n,cl}$ given by
\ben
F^1 = \Vect \ltimes \hOmega^1_{n,cl} \supset F^0 = \hOmega^1_{n,cl} .
\een
\owen{This isn't a subcategory as a filtration is data.}
Let $M$ be such a module. Taking the associated graded with respect to
this filtration we obtain a graded module ${\rm gr} \; M$ for the pair
$(\Vect, \GL_n)$. 

\begin{eg}
We have the formal Atiyah sequence
\be\label{broidses}
0 \to \hO_n \to \Tilde{\cT}_n \to \hT_n \to 0,
\ee
where $\Tilde{\cT}_n$ is...\ryan{Finish...this sequence is corresponds to $\id_{\Omega^1_{n}} \in C^1_{Lie} (\Vect \rtimes \hOmega^1_{n,cl} , \hOmega^1_{n,cl})$}. Note that by construction this can be considered as a sequence of $(\Vect
\ltimes \hOmega^1_{n,cl} , \GL_n)$-modules.
\end{eg}




\subsection{TDO's from descent}\label{sect:TDODesc} 

\subsubsection{Classical theory of TDO's}

We briefly recall a construction of locally trivial TDO's. A standard reference is \cite{Ginzburg} particularly section 2.2 or appendix A of \cite{Kazhdan}.


For any $\alpha \in H^1 (X, \Omega^1_{cl})$ we have an associated Atiyah algebra
\ben
0 \to \cO_X \to \cT_X^\alpha \to \cT_X \to 0 .
\een
Up to isomorphism, this association is a bijection, i.e., Atiyah algebras are classified by $H^1 (X, \Omega^1_{cl})$.

\begin{dfn} 
The {\em twisted differential operators} $TDO^\alpha_X$ corresponding to an element $\alpha \in {\rm H}^1(X ; \Omega^1_{cl})$ is the sheaf of algebras~$\cU_{\cO_X} (\cT^\alpha_X)$.
\end{dfn}

\subsubsection{}

Recall the formal Atiyah sequence
\be\label{broidses}
0 \to \hO_n \to \Tilde{\cT}_n \to \hT_n \to 0.
\ee



\begin{prop} Fix a GK-structure $\sigma$ on $X$. There is an isomorphism of Lie algebroids on $X$
\ben
\sdesc(\sigma, \alpha, \Tilde{\cT}_n) \cong \cT^\alpha_X .
\een 
\end{prop}
\begin{proof} 
Consider the short exact sequence (\ref{broidses}) of $(\Vect
\ltimes \hOmega^1_{n,cl} , \GL_n)$ modules. It is determined by the
cocycle $\id_{\Omega^1_{n}}$ which maps under the characteristic map
  to $(\Fr_X, \omega^\sigma_\alpha)$
  to the element $\alpha \in {\rm H}^1(X ; \Omega^1_{X,cl})$. Thus we
  see that when we apply descent to the short exact sequence of
  $(\Vect \ltimes \hOmega, \GL_n)$-modules we get a short exact
  sequence of $\cO_X$-modules 
\ben
0 \to \cO_X \to \sdesc(\sigma, \alpha, \Tilde{\cT}_n) \to \cT_X \to 0,
\een
which is the Atiyah algebra classified by $\alpha$. We conclude that as $\cO_X$-modules
$\sdesc(\sigma, \alpha, \Tilde{\cT}_n) \cong \cT_X^\alpha$. Moreover, by the classification of Atiyah algebras, it is an equivalence as Lie algebroids.
\end{proof}


It then follows from Proposition \ref{prop:enveloping} that we obtain twisted differential operators via descent of the formal Weyl algebra.

\begin{cor}\label{cor:descTDO}
Fix a GK-structure $\sigma$ and let $\alpha \in {\rm H}^1(X, \Omega^1_{cl,X})$. 
Then 
\[
\sdesc(\sigma, \alpha, \hA_n) \cong TDO^\alpha_{X}.
\]
That is, the Gelfand-Kazhdan descent of the completed Weyl algebra $\hA_n$ 
along the flat $(\Vect \ltimes \hOmega^1_{n,cl}, \GL_n)$-bundle $(\Fr_X, \omega_\alpha^\sigma)$ 
recovers the sheaf of TDO's on $X$ twisted by~$\alpha$.
\end{cor}


